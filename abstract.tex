\chapter*{Abstract}

Today, millions of users follow their teams' games online to keep up-to-date regarding the events of a match. Some of those had a special connection to their hometown team, but since it plays in way lower leagues and without much exposure,  the users end up missing information and losing the passion they once had for the hometown team. There is, however,  a specific group of users that keeps following the games of the smaller teams, and most importantly: sharing updates about them. 

Platforms that allow game commentary sharing enable the most passionate fans who still watch the smaller leagues to share what is going on in the game, report the events, and build the game's history, totally community-driven. 

Tools for this already exist but are either somewhat outdated or do not completely let the users get involved in the commentary, restricting them to a more passive role, hence the opportunity to build something better. 

This work involved creating a Real-Time web application that allows users to publish event updates for any event, increasing their connection to the lower leagues as it already happens in the upper echelons. Additionally, as stadiums internet connectivity is often poor, the application allows users to post while offline, syncing when the internet comes back, and featuring a conflict resolution functionality to improve user experience (UX). Early research points towards a positive influence of automatic conflict resolution on the user experience, even though conflicts were not many during the experiments. All inquired users responded positively to the application, mentioning that they would use the application again, after their first usage, and journalists using the tool to cover official matches professionally collectively agreed that it represents an user experience boost when compared to the existing tool. 


\vspace*{10mm}\noindent
\textbf{Keywords}: Real-Time, Web, Sports, Conflict Resolution, Reputation systems

\chapter*{Resumo}

Hoje em dia milhões de utilizadores mantêm-se a par dos jogos das suas equipas online, de forma a estar a par dos eventos ao longo destes, caso não possam assistir diretamente. Alguns tinham uma ligação especial à equipa local, mas jogando em ligas de escalão inferior sem grande cobertura televisiva ou até mesmo jornalstica, acabam por perder interesse e a paixão que sentiam outrora. Ainda assim, ainda existe um grupo de utilizadores que continua a seguir os jogos das equipas mais pequenas, e mais importante que isso: a partilhar atualizações sobre estes. 

Plataformas que permitam o comentário ao longo dum jogo permitem aos fãs mais apaixonados que ainda seguem as equipas locais partilhar o que vai acontecendo ao longo do evento, reportando o que se passa e construindo a história do jogo, num esforço totalmente comunitário. 

Ferramentas como estas já existem, contudo estão obsoletas, ou não permitem completamente aos utilizadores o envolvimento no comentário do jogo, restringindoos a um papel mais passivo, daí a oportunidade de criar algo único e inovador. 

Este trabalho resultou na criação de uma Aplicação Web que permite aos utilizadores a partilha de atualizações de um evento desportivo em tempo-real, aumentando a ligação às ligas mais inferiores, da mesma forma que acontece com os escalões superiores. Adicionalmente, e uma vez que a ligação à internet nos estádios é geralmente instável, a aplicação permite aos utilizadores interagir enquanto estão em modo offline, sincronizando assim que a ligação seja restabelecida, incluindo uma funcionalidade de resolução de conflitos de forma a melhorar a experiencia do utilizador. Resultados iniciais demonstram que a resolução de conflitos automática tem influência positiva na experiencia de utilizador, ainda que o numero de conflitos não tenha sido significativo. Todos os utilizadores questionados responderam positivamente em relação à aplicação, afirmando que a usariam novamente após o primeiro uso. Jornalistas que usam a ferramenta para relatar jogos oficiais de forma profissional concordaram no facto desta aplicação representar um incremento significativo em termos de experiência de utilizador, quando comparada com a ferramenta anterior. 


\vspace*{10mm}\noindent
