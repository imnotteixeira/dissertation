\chapter{Problem Statement}\label{chap:problem}

\section*{}

This chapter describes the problem tackled in this dissertation, including the planned features, the expected result and the planning to achieve it. Section \ref{sec:prob-def} describes the features to be developed. Section \ref{sec:planning} addresses the planning of the intended tasks. The planned development methodology is defined in Section \ref{sec:methodology}. Some preliminary ideas on the resolution of parts of the problem will be presented in the next chapter (Chapter \ref{chap:problem-solution}). 

\section{Problem Definition}\label{sec:prob-def}

TODO
US here

\section{Planning}\label{sec:prob-planning}

TODO
GANTT here

\section{Methodology}\label{sec:prob-methodology}

TODO
mention scrum like stuff, in order to be able to gather feedback with ready products in between sprints (vs kanban which would be more continuous)


This work is intended to be tested on real users during its development. Thus, an agile methodology seems to be more fitting than a waterfall approach \cite{beck2001agile}. The most common are Scrum, XP and Kanban.

