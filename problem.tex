\chapter{Problem Statement}\label{chap:problem}

\section*{}

This chapter describes the problem tackled in this work, including the planned features and the expected result. Section~\ref{sec:key-concepts} provides some key concepts used throughout the document, Section~\ref{sec:prob-def} presents the features to be developed in User Story format, some additional technical requirements are presented in Section~\ref{sec:tech-reqs}. Later, in Section~\ref{sec-hyposthesis}, this work's hypothesis is presented and in Section~\ref{sec:research-questions}, some research questions are specified, for which this work aims to provide answers.

The implemented solution for this problem is presented in the next chapter.  

\section{Key Domain Concepts}\label{sec:key-concepts}
There are some key concepts that will be mentioned throughout this document which are relevant to distinguish, regarding the platforms related to this work.
\begin{description}
    \item[ZeroZero] the ZeroZero website (zerozero.pt), which currently is used by users to follow and by journalists to cover matches;
    \item[ZeroZero API] the API that powers the ZeroZero website; it provides teams and players information, as well as match details, so it will also be used by ZeroZero Live;
    \item[ZeroZero Live] used interchangeably with zerozero.live --- the domain through which it can be accessed --- is the new application, to be developed in this work.
\end{description}

\section{Problem Definition}\label{sec:prob-def}

The project's goal is to develop a web application that allows users to follow a sporting event in a real-time chat-like environment where everyone can input game events. For users that are just following, it would work as live coverage of the event; for contributors, it should be resilient to network failures due to stadiums' Wi-Fi limitations.
Due to the above goal, some necessary features start to surface, such as real-time conflict resolution and the inherent reputation system for tie-breaking when necessary. 

A prototype that allows event submission is already available, and since it is using React, which is an appropriate technology for this task, it will be kept. Some features still need to be polished, but most of the UI is already done, which will allow a bigger focus on the actual real-time conflict resolution problem.

The features of the prototype are described as follows, in User Story format\footnote{In software development and product management, a user story is an informal, natural language description of one or more features of a software system. User stories are often written from the perspective of an end-user or user of a system.}:

\begin{enumerate}[leftmargin  = 3.25\parindent, align=left, label=US\arabic*, start=1]
    \item As a user, I want to be able to join a sport event channel, where I can see details about the event in real-time (either pre-filled or contributed by other users), so that I have information about what is happening in the event.
    \item As a user, I want to be able to be able to post event updates while on an event channel, in a chat-like interaction, so that I can easily contribute in an input experience I recognize.
    \begin{itemize}
        \item Event updates include: Starting players, Goals, Fouls, Set-Pieces, Cards, Substitutions, Game-Time information, and generic match information
    \end{itemize} 
    \item As a user, I want to be able to use the application while in offline mode, and have it synchronize once the connection is resumed, so that I don't lose information nor focus when my connection drops.
    \item As a user, I want to see a value representing the reputation of other users in a given event channel, so that I have a basis on which to decide if I trust them. 
    \item As a user, I want to be able to delete inputted events, so that I can let others know that they might not be true and manifest my intention to change it.
    \item As a user, I want to be able to see if there are any pending conflicts to be resolved, so that I can clearly see if I need to solve any conflicting information.
    \item As a user, I want to be able to resolve any pending conflicts, so that I can keep the event's history clean and understandable.
    \item As a user, I want to be able to join an event channel mid-session, being able to see the previous information, so that I have more flexibility, not losing context if I arrive some time late.
    \item\label{user-story:sync-to-api} As a user, I want to be able to see the events generated on the ZeroZero platform (not ZeroZero Live), so that I can follow the game even if don't want to comment about it.
    \item\label{user-story:sync-from-api} As an official ZeroZero reporter, I want to be able to input events on the old platform, and have them synced to ZeroZero Live, so that I can use other tooling that depends on it to provide statistics, and still be able to comment manually on ZeroZero Live.
\end{enumerate}

\section{Technical Requirements}\label{sec:tech-reqs}

Apart from the functional requirements above, the tool should adhere to the following technical requirements:

\begin{enumerate}[leftmargin  = 3.25\parindent, align=left, label=TR\arabic*, start=1]
    \item The system should use the authentication API from ZeroZero in order to authenticate users and provide session functionality;
    \item Changes and updates to the application should not require total outages of service;
    \item There will be sufficient logging to quickly identify system problems;
    \item The tool should be available as a web application, able to work on any browser (desktop, tablet or mobile);
    \item The application may require an internet connection on setup, but should be resilient to network failures mid-session;
    \item The system should be horizontally scalable as the number of users grows;
\end{enumerate}

\section{Hypothesis}\label{sec-hyposthesis}

This work is done to test the following hypothesis:

\begin{quote}
    \say{\textit{A conflict-resolution system improves the user experience in a real-time multi-user crowdsourcing environment.}}
\end{quote}

User Experience can mean many things. In this work, it will be coupled with satisfaction of use, and willingness to keep using the product. In order to verify this hypothesis, an existing prototype for the ZeroZero Live web application was extended in order to include the real-time, offline tolerance, and conflict resolution features. By exposing this version to journalists and common users, it was possible to measure their interactions and how they felt with regards to the application. That analysis is present in Chapter~\ref{chap:evaluation-and-analysis}.

\section{Research Questions}\label{sec:research-questions}

Given the hypothesis proposed above and the problem definition provided above (Section~\ref{sec:prob-def}), the main research question of this work is:

\begin{enumerate}[leftmargin  = 3.25\parindent, align=left]
    \item[RQ] Does an high number of conflicts impact the User Experience in a collaborative web platform?
\end{enumerate}

Additionally, an since this question is too generic, there are a few more research questions: 
\begin{enumerate}[leftmargin  = 3.25\parindent, align=left, label=RQ\arabic*, start=1]
    \item How is the number of inputs related to the number of conflicts?
    \item How is the number of collaborators related to the number of conflicts?
\end{enumerate}

\section{Summary}

In summary, we intend to research the impact of conflict resolution on a real-time, offline-tolerant crowdsourcing environment. To verify that, a multi-user real-time application allowing the coverage of sporting events with offline tolerance will be developed. That entails the existence of conflict resolution strategies, as well as a reputation system, which should take time into consideration, making reputation more dynamic. All of this should be integrable with the existing ZeroZero system, and use modern strategies to solve the technological challenges.

