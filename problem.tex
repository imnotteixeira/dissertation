\chapter{Problem Statement}\label{chap:problem}

\section*{}

This chapter describes the problem tackled in this dissertation, including the planned features, the expected result and the planning to achieve it. Section \ref{sec:prob-def} describes the features to be developed. Section \ref{sec:prob-planning} addresses the planning of the intended tasks. The planned development methodology is defined in Section \ref{sec:prob-methodology}. Some preliminary ideas on the resolution of parts of the problem will be presented in the next chapter (Chapter \ref{chap:problem-solution}). 

\section{Problem Definition}\label{sec:prob-def}

As mentioned in the Introduction (Chapter \ref{chap:intro}), the goal of the project is to develop a web application that allows users to follow a sport event in a real-time chat-like environment where everyone can input game events. For users that are just following, it would work like a live coverage of the event, for contributors it should be resilient to network failures, due to the Wi-Fi limitations of stadiums.

Due to the above goal, some necessary features start to surface such as real-time conflict resolution and the inherent reputation system for tie-breaking when necessary. 

A prototype that allows event submission is already available, and since it is using React, and it is an appropriate technology for this task, it will be kept. Some features still need to be polished, but most of the UI is already done, which will allow a bigger focus on the actual real-time conflict resolution problem.

The features are described as follows, in User Story format\footnote{In software development and product management, a user story is an informal, natural language description of one or more features of a software system. User stories are often written from the perspective of an end user or user of a system.}:

\begin{itemize}[leftmargin  = 3.25\parindent, align=left]
    \item[US01] User story 1
    \item[US02] User story 2
    \item[US03] User story 3
    \item[US03asdadasds] User story 3
\end{itemize}

TODO
US here

\section{Planning}\label{sec:prob-planning}

TODO
GANTT here

\section{Methodology}\label{sec:prob-methodology}

TODO
mention scrum like stuff, in order to be able to gather feedback with ready products in between sprints (vs kanban which would be more continuous)


This work is intended to be tested on real users during its development. Thus, an agile methodology seems to be more fitting than a waterfall approach \cite{beck2001agile}. The most common are Scrum, XP and Kanban.

