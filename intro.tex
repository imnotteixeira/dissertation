\chapter{Introduction} \label{chap:intro}

\section*{}

{\Huge TODO} \\


Today, millions of users follow their teams' games online to keep up-to-date regarding the events of a match \footnote{https://www.facebook.com/business/news/insights/the-changing-profile-of-sports-fans-around-the-world}. Some of those had a special connection to their hometown team, but since they play in way lower leagues and without much exposure, the users end up missing information and losing the passion they once had for the hometown team.

There is, however,  a specific group of users that keeps following the games of the smaller teams, and most importantly: sharing updates about them. One platform that allows users to do that, as of today, is zerozero.pt, from ZOS. This enables the most passionate fans who still watch the smaller leagues to share what is going on in the game, report the events, and build the game's history, totally community-driven. This tool exists and is somewhat outdated, hence the opportunity to build something better, including automatic conflict resolution, offline mode and a mobile-friendly interface out of the box.

\section{Goals}
The goal is to allow multiple users to report the incidents in a sporting event, which will show up for everyone following that match in real-time. As internet connectivity is often poor inside stadiums, the tool must allow offline work, synced whenever possible. This can generate many data inconsistencies, which must be handled by the tool.

Parallel to this, we want to provide the best possible User Experience, since inconsistencies can seem confusing for users. For this, we indend to measure and test different alternatives, in order to elicit what is the desired experience. 

This project will provide an approach to this problem, and the following sections provide more details on the project's key objectives.

\section{Offline Availability} \label{sec:offline-avail-intro}

As previously stated, internet connection in stadiums is poor most of the time. Thus, the users must have the option to interact with the application and synchronize once possible. This will obviously lead to data consistency issues (i.e.,\ two users report a goal, changing the result to \say{1-0} for example, but one of them is offline, so when it finally synchronizes, the result is already \say{3-2}, and it should not be overwritten.)

More information on this topic is presented in Section~\ref{sec:offline-avail-sota} and a proposed solution will be stated in Section~\ref{sec:prob-solution-offline-avail}.

\section{Conflict Resolution} \label{sec:conflict-res-intro}

Another objective of the tool is to provide users with automatic conflict resolution when possible. Some strategies are depicted in the \textit{State of the Art} section, in Chapter~\ref{sec:conflict-res-sota}. Here, it is important to preserve the truth and the most up-to-date versions of data. In this scenario, there might not be a source of truth present to verify and validate all inputs, so other strategies must be used, such as agreement-based implicit voting --- if nobody questions a user's input, it must be true until stated otherwise.

Additionally, the tool can use different strategies to solve conflicts automatically, thus improving the user experience (UX). More on this topic is available in Section~\ref{sec:conflict-res-sota} and a preliminary proposed solution can be found in Section~\ref{sec:prob-solution-conflict-res}.

\section{Reputation System} \label{sec:rep-sys-intro}

The third key-objective of the application will be the reputation system. Currently, there already exists a ranking concept, and a \say{trusted} user, which is the equivalent to the maximum reputation and should be considered the source of truth in case of conflict.

But what about the cases where two \say{non-trusted} users' inputs conflict, or even the case of two \say{trusted} users? Who should win? To resolve conflicts, an answer to these \textit{conundrums} is fundamental. Ergo a new reputation system is required, and more details are available in Section~\ref{sec:rep-sys-sota}. A preliminary proposed solution is presented in Section~\ref{sec:problem-solution-rep-sys}.

\section{Document Structure}

In Chapter~\ref{chap:sota}, a comparison with a similar project is made, as well as a \textit{State of the Art} exploration on the multiple scopes of this project. Chapter~\ref{chap:problem} defines the problem and proposes solutions for it, which are planned in Chapter~\ref{chap:problem-planning} Conclusions are present in Chapter~\ref{chap:concl}. 