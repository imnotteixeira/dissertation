\chapter*{Abstract}

As of today, millions of users follow their teams' games online to keep up-to-date regarding the events of a match. Some of those had a special connection to their hometown team, but since they play in way lower leagues and without much exposure, oftentimes the users end up missing information and losing the passion they once had for the hometown team. There is a specific group of users, however, that keeps following the games of the smaller teams, and most importantly: sharing updates about them. Platforms that allow game commentary sharing enable the most passionate fans that still watch the smaller leagues to share what is going on in the game, reporting the events and building the game's history, totally community-driven. Tools for this already exist but are either somewhat outdated, or do not completely let the users get involved in the commentary, restricting them to a more passive role, hence the opportunity to build something better. This work involves the creation of a Real-Time web application that allows users to publish event updates for any event, increasing their connection to the lower leagues as it already happens in the upper echelons. Additionally, as stadiums internet connectivity is often poor, the application will allow users to post while offline, syncing when the internet comes back, and featuring a conflict resolution functionality to improve user experience (UX). A proposed solution is presented in this document, as well as a plan for its development.


\vspace*{10mm}\noindent
\textbf{Keywords}: Real-Time, Web, Sports, Conflict Resolution, Reputation systems

\chapter*{Resumo}

Hoje em dia milhões de utilizadores mantêm-se a par dos jogos das suas equipas online, de forma a estar a par dos eventos ao longo destes, caso não possam assistir diretamente. Alguns tinham uma ligação especial à equipa local, mas jogando em ligas de escalão inferior sem grande cobertura televisiva ou até mesmo jornalstica, acabam por perder interesse e a paixão que sentiam outrora. Ainda assim, ainda existe um grupo de utilizadores que continua a seguir os jogos das equipas mais pequenas, e mais importante que isso: a partilhar atualizações sobre estes. Plataformas que permitam o comentário ao longo dum jogo permitem aos fãs mais apaixonados que ainda seguem as equipas locais partilhar o que vai acontecendo ao longo do evento, reportando o que se passa e construindo a história do jogo, num esforço totalmente comunitário. Ferramentas como estas já existem, contudo estão obsoletas, ou não permitem completamente aos utilizadores o envolvimento no comentário do jogo, restringindoos a um papel mais passivo, daí a oportunidade de criar algo único e inovador. Este trabalho involve a criação de uma Aplicação Web que permita aos utilizadores a partilha de atualizações de um evento desportivo em tempo-real, aumentando a ligação às ligas mais inferiores, como acontece com os escalões superiores. Adicionalmente, e uma vez que a ligação à internet nos estádios é geralmente instável, a aplicação deverá permitir aos utilizadores interagir enquanto estiverem em modo offline, sincronizando assim que a ligação seja restabelecida, incluindo uma funcionalidade de resolução de conflitos de forma a melhorar a experiencia do utilizador. Este documento apresenta uma proposta de solução, bem como um plano para a sua implementação.


\vspace*{10mm}\noindent
\textbf{Keywords}: Real-Time, Web, Sports, Conflict Resolution, Reputation systems
